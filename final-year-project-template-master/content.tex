%!TEX root = project.tex

\chapter*{About this project}
 % A brief description of what the project is, in about two-hundred and fifty words.
 
 Welcome to my minor Dissertation my name is Tomás O'Malley (G00361128) and I am a final year student studying  a Bsc Honours in Software Development @ Galway Mayo Institute of Technology. After my years of study and my deep interest in technology and how we interface device day-to-day I settled on developing a social network for local communities.What is a social network ?\\ \\ 
 
 A network of social interactions and personal relationships , common examples of social media platform include Facebook , Instagram and twitter. Smartphones have become an addition to how humans socialize and congregate, my applications target audience are people who want to congregate in local communities and share their photos , experiences and safely create events in their chosen communities with ease using their smart device. \\
 
 Each component from the back-end written in the secure language  of ruby on rails to the database system mongodb.This document outlines in great detail the many system design/architecture.Tackling a modern architecture will challenge and award when in an industrial environment.\\
 
 The application is focused around users such as rookies and administrators.User can use the forums to upload documents , private message each other and use  built in schedules to manage a healthy online and offline social life.\\
 
 This application incorporates some of the juggernauts  areas of modern day computing such as databases , web frameworks and mobile devices.All of the programming will be stored and document using the version control Git at \href{https://github.com/OmalleyTomas98/MinorDissertation}{Github} \\
 
 


\paragraph{Author}
 % Explain here who the authors are.
  The whole project was developed by Tomás O'Malley (G00361128) all project work and material can be found using my \href{https://github.com/OmalleyTomas98/MinorDissertation}{GitHub} \\The project was developed using an agile methodology to make sure I deliver a crucial component on time.
 You can find more about me via my Online College Portfolio  \href{https://omalleytomas98.github.io}{Here}


\chapter{Introduction}
 % The introduction should be about three to five pages long.
% Make sure you use references~\cite{einstein}

Hello my name is Tomás O'Malley and I am a fourth year Galway student studying Software Development.It is mandatory for completion of our Bsc Honours in Software to deliver a Final Year Project and Dissertation  to ensure we have a tight grasp on computing before continuing our studies or working in a professional environment.\\

\\During our College Introduction during week 1 Semester 7 I decided to begin brainstorming for both modern technologies and the implications of these technologies in modern day.The Project needed to be delivered/developed starting from week 1 and completed by the end of semester 2.\\

It was crucial to begin planning once I was delegated a project supervisor (Dr John French) for the module "Applied Project and Minor Dissertation".





\chapter{Methodology}
\begin{itemize}
\item %  Provide a context for your project.
The aim for my 2020/2021 FYP is to develop a mobile  Social Media Platform for local communities.
\item % Set out the objectives of the project
Objectives of project. (1) Create a social media application that can be used on mobile devices.(2) Research and learn the Ruby on rails language. (3) Learn the security protocols and incorporate  a strong back-end to safely store medias.(3) Create a secure and well defined database system using mongodb database systems. (4) Develop a clear understanding of User e
\item Briefly list each chapter / section and provide a 1-2 line description of what each section contains.
\item List the resource URL (GitHub address) for the project and provide a brief list of the main elements at the URL.
\end{itemize}



\chapter{Technology Review}
About one to two pages.
Describe the way you went about your project:
\begin{itemize}
\item  % Agile / incremental and iterative approach to development. Planning, meetings.
To make sure elements of the project were developed on time and at a level of quality I decided to use GitHub version control to manage my code base while adapting an agile approach by delegating tasks to myself and developing each component in a set time.Each week I arranged a time  with my project supervisor @ 10am Friday mornings where I delivered a LaTeX write up disclosing the many additions and obstacles of the project.


\item %  What about validation and testing? Junit or some other framework.

Automated testing will be in place to test the accuracy and responsiveness.


\item % If team based, did you use GitHub during the development process.

The whole project was developed by myself and Git Version control was incorporated into the project to make sure I timed the sprints 


\item % Selection criteria for algorithms, languages, platforms and technolo-gies.



\begin{enumerate}
  \item algorithms : Sorting algorithms bubble sort
  \item languages  : Ruby on rails , JavaScript , React Framework , noSQL.  
\item platforms  : Android Platform

\item Technologies   : Modern Social Media 

\end{enumerate}

\end{itemize}

\chapter{System Design}
 About seven to ten pages.
 
 
 
 log in system architecture 
 
 
 
 
 
\begin{itemize}
\item Describe each of the technologies you used at a conceptual level. Standards, Database Model (e.g. MongoDB, CouchDB), XMl, WSDL, JSON, JAXP.
\item Use references (IEEE format, e.g. [1]), Books, Papers, URLs (timestamp) – sources should be authoritative. 
\end{itemize}


\chapter{System Evaluation}
As many pages as needed.




\begin{itemize}
\item Architecture, UML etc. An overview of the different components of the system. Diagrams etc… Screen shots etc.






\end{itemize}


\chapter{Conclusion}
As many pages as needed.
\begin{itemize}
\item Prove that your software is robust. How? Testing etc. 
\item Use performance benchmarks (space and time) if algorithmic.
\item Measure the outcomes / outputs of your system / software against the objectives from the Introduction.
\item Highlight any limitations or opportuni-ties in your approach or technologies used.
\end{itemize}

\chapter{References}
About three pages.

\begin{itemize}
\item Briefly summarise your context and ob-jectives (a few lines).
\item Highlight your findings from the evalua-tion section / chapter and any opportuni-ties identified.
\end{itemize}

\chapter{Appendices }


